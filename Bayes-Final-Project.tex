% Options for packages loaded elsewhere
\PassOptionsToPackage{unicode}{hyperref}
\PassOptionsToPackage{hyphens}{url}
%
\documentclass[
]{article}
\usepackage{lmodern}
\usepackage{amssymb,amsmath}
\usepackage{ifxetex,ifluatex}
\ifnum 0\ifxetex 1\fi\ifluatex 1\fi=0 % if pdftex
  \usepackage[T1]{fontenc}
  \usepackage[utf8]{inputenc}
  \usepackage{textcomp} % provide euro and other symbols
\else % if luatex or xetex
  \usepackage{unicode-math}
  \defaultfontfeatures{Scale=MatchLowercase}
  \defaultfontfeatures[\rmfamily]{Ligatures=TeX,Scale=1}
\fi
% Use upquote if available, for straight quotes in verbatim environments
\IfFileExists{upquote.sty}{\usepackage{upquote}}{}
\IfFileExists{microtype.sty}{% use microtype if available
  \usepackage[]{microtype}
  \UseMicrotypeSet[protrusion]{basicmath} % disable protrusion for tt fonts
}{}
\makeatletter
\@ifundefined{KOMAClassName}{% if non-KOMA class
  \IfFileExists{parskip.sty}{%
    \usepackage{parskip}
  }{% else
    \setlength{\parindent}{0pt}
    \setlength{\parskip}{6pt plus 2pt minus 1pt}}
}{% if KOMA class
  \KOMAoptions{parskip=half}}
\makeatother
\usepackage{xcolor}
\IfFileExists{xurl.sty}{\usepackage{xurl}}{} % add URL line breaks if available
\IfFileExists{bookmark.sty}{\usepackage{bookmark}}{\usepackage{hyperref}}
\hypersetup{
  pdftitle={Final Project},
  hidelinks,
  pdfcreator={LaTeX via pandoc}}
\urlstyle{same} % disable monospaced font for URLs
\usepackage[margin=1in]{geometry}
\usepackage{graphicx,grffile}
\makeatletter
\def\maxwidth{\ifdim\Gin@nat@width>\linewidth\linewidth\else\Gin@nat@width\fi}
\def\maxheight{\ifdim\Gin@nat@height>\textheight\textheight\else\Gin@nat@height\fi}
\makeatother
% Scale images if necessary, so that they will not overflow the page
% margins by default, and it is still possible to overwrite the defaults
% using explicit options in \includegraphics[width, height, ...]{}
\setkeys{Gin}{width=\maxwidth,height=\maxheight,keepaspectratio}
% Set default figure placement to htbp
\makeatletter
\def\fps@figure{htbp}
\makeatother
\setlength{\emergencystretch}{3em} % prevent overfull lines
\providecommand{\tightlist}{%
  \setlength{\itemsep}{0pt}\setlength{\parskip}{0pt}}
\setcounter{secnumdepth}{-\maxdimen} % remove section numbering

\title{Final Project}
\author{}
\date{\vspace{-2.5em}}

\begin{document}
\maketitle

\hypertarget{particle-filtering}{%
\section{Particle Filtering}\label{particle-filtering}}

Particle filtering is a sequential Monte-Carlo (MC) method that seeks to
predict a hidden state variable (\(\textbf{x}\)) from a series of
observations (\(\textbf{y}\)). \(p(\textbf{x}_0)\) is the initial state
of the distribution, the transition equation is
\(p(\textbf{x}_t|\textbf{x}_{t-1})\), and
\(p(\textbf{y}_t|\textbf{x}_t)\) is the marginal distribution of the
observation. Using Bayes' theorem, we derive an expression for
\(p(\textbf{x}_{0:t}|\textbf{y}_{1:t})\), the marginal distribution of
the hidden state variable from the observations:

\[p(\textbf{x}_{0:t}|\textbf{y}_{1:t}) = \frac{p(\textbf{y}_{t}|\textbf{x}_{t})p(\textbf{x}_{t}|\textbf{y}_{1:t-1})}{\int{p(\textbf{y}_{t}|\textbf{x}_{t})p(\textbf{x}_{t}|\textbf{y}_{1:t-1})}d\textbf{x}_t}\]
\[p(\textbf{x}_{0:t}|\textbf{y}_{1:t}) \propto p(\textbf{y}_{t}|\textbf{x}_{t})p(\textbf{x}_{t}|\textbf{y}_{1:t-1})\]
We can also compute \(p(\textbf{x}_{t}|\textbf{y}_{1:t})\) recursively
via the marginal distribution:

\[p(\textbf{x}_{t}|\textbf{y}_{1:t}) = \int {p(\textbf{x}_{t}|\textbf{x}_{t-1})p(\textbf{x}_{t-1}|\textbf{y}_{1:t-1})}d \textbf{x}_{t-1}\]
To find the expected value of \(E[f(x_t)]\):

\[E[f(\textbf{x}_t)] = \int f(\textbf{x}_{0:t}) p(\textbf{x}_{0:t}|\textbf{y}_{1:t})d \textbf{x}_{0:t}\]
Do we need intermediate steps here?

\[E[f(\textbf{x}_t)] = \frac{\int f(\textbf{x}_{0:t}) p(\textbf{x}_{0:t}|\textbf{y}_{1:t})d \textbf{x}_{0:t}}{\int p(\textbf{x}_{0:t}|\textbf{y}_{1:t})d \textbf{x}_{0:t}}\]

To evalute this integral, we introduce \(w(x_{0:t})\), the importance
weight. The importance weight is equal to:

\[w(x_{0:t}) = \frac{p(x_{0:t}|y_{1:t})}{\pi(x_{0:t}|y_{1:t})}\]

the importance sampling factor. The weight is very important in a
particle filter algorithm as it allows us to pick which states are more
likely than others and reduce down potential states before resampling.
This weight, and subsequently, the importance sampling factor relies on
(in our project with crude oil prices) the probability of a tomorrow's
future price, given today's spot price divided by 𝛑, which is denoted by
a factor that assesses different variables that influence the movement
of tomorrow's future price.

\hypertarget{dataset}{%
\section{Dataset}\label{dataset}}

\hypertarget{bibliography}{%
\section{Bibliography}\label{bibliography}}

\[x\]

\end{document}
